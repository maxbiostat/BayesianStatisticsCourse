\documentclass[9pt]{beamer}
\usepackage{amsmath, amssymb, amsthm, mathtools, graphicx, float, subfigure, booktabs, enumitem}
\usepackage{hyperref}
\urlstyle{same}
\usepackage{minted}
\usepackage{pifont}
\usepackage{xcolor}
\usepackage[utf8]{inputenc} % usually not needed (loaded by default)
\usepackage[T1]{fontenc}
\hypersetup{colorlinks=true,citecolor=blue}
\usepackage{tikz}
\usepackage{fontawesome}
\usepackage{libertine}
\usepackage[libertine]{newtxmath}
\usetikzlibrary{calc,shapes}
\usepackage[normalem]{ulem}
\setbeamertemplate{theorems}[numbered]
\usepackage[authoryear,round]{natbib}
% \usepackage[portuguese]{babel}
\usetheme[pageofpages=of,% String used between the current page and the
                         % total page count.
          bullet=circle,% Use circles instead of squares for bullets.
          titleline=true,% Show a line below the frame title.
          alternativetitlepage=true,% Use the fancy title page.
          %titlepagelogo=logo-fiocruz,% Logo for the first page.
          %watermark=watermark-poli
          to,% Watermark used in every page.
          %watermarkheight=100px,% Height of the watermark.
          %watermarkheightmult=4,% The watermark image is 4 times bigger
                                % than watermarkheight.
          ]{Torino}
\usecolortheme{freewilly}          
%%%% Box options
\newcommand{\tikzmark}[1]{\tikz[overlay,remember picture] \node (#1) {};}
%%%% Background settings          
% \setbeamercolor{normal text}{fg=white,bg=black!90}
% \setbeamercolor{structure}{fg=white}
% \setbeamercolor{alerted text}{fg=red!85!black}
% \setbeamercolor{item projected}{use=item,fg=black,bg=item.fg!95}
% \setbeamercolor*{palette primary}{use=structure,fg=structure.fg}
% \setbeamercolor*{palette secondary}{use=structure,fg=structure.fg!95!black}
% \setbeamercolor*{palette tertiary}{use=structure,fg=structure.fg!90!black}
% \setbeamercolor*{palette quaternary}{use=structure,fg=structure.fg!95!black,bg=black!80}
% \setbeamercolor{title}{fg=white}
% \setbeamercolor{frametitle}{bg=white}
% \setbeamercolor*{framesubtitle}{fg=white}
% \setbeamercolor*{block title}{parent=structure,bg=black!95}
% \setbeamercolor*{block body}{fg=black,bg=black!10}
% \setbeamercolor*{block title alerted}{parent=alerted text,bg=black!95}
% \setbeamercolor*{block title example}{parent=example text,bg=black!95}


%%%% Maths crap
\newtheorem{remark}{Remark}[]
\newtheorem{theo}{Theorem}[]
\newtheorem{exercise}{Exercise}[]
\newtheorem{defn}{Definition}[]
\newtheorem{question}{Question}[]
\newtheorem{idea}{Idea}[]
\newtheorem{property}{Property}[]
%%%% Itemize settings 
\setlist[itemize,1]{label=$\bullet$}
\setlist[itemize,2]{label=$\diamond$}

% \setbeamercolor{block title}{use=structure,fg=white,bg=structure.fg!75!black}
% \setbeamercolor{block body}{parent=normal text,use=block title,bg=block title.bg!10!bg}

\setbeamercolor{block title}{use=structure,fg=white,bg=black}
\setbeamercolor{block body}{parent=normal text,use=block title,fg=white,bg=gray}
\setbeamercolor{frametitle}{bg=black, fg=white}

%%%%%%%%%%%%%%%%%%%% Notation stuff
\newcommand{\indep}{\perp \!\!\! \perp} %% indepence
\newcommand{\pr}{\operatorname{Pr}} %% probability
\newcommand{\vr}{\operatorname{Var}} %% variance
\newcommand{\rs}{X_1, X_2, \ldots, X_n} %%  random sample
\newcommand{\irs}{X_1, X_2, \ldots} %% infinite random sample
\newcommand{\rsd}{x_1, x_2, \ldots, x_n} %%  random sample, realised
\newcommand{\Sm}{\bar{X}_n} %%  sample mean, random variable
\newcommand{\sm}{\bar{x}_n} %%  sample mean, realised
\newcommand{\Sv}{\bar{S}^2_n} %%  sample variance, random variable
\newcommand{\sv}{\bar{s}^2_n} %%  sample variance, realised
\newcommand{\bX}{\boldsymbol{X}} %%  random sample, contracted form (bold)
\newcommand{\bx}{\boldsymbol{x}} %%  random sample, realised, contracted form (bold)
\newcommand{\bT}{\boldsymbol{T}} %%  Statistic, vector form (bold)
\newcommand{\bt}{\boldsymbol{t}} %%  Statistic, realised, vector form (bold)
\newcommand{\mle}{\hat{\theta}_{\text{MLE}}}
\newcommand{\mb}{\hat{\theta}_{\text{B}}}
\newcommand{\map}{\hat{\theta}_{\text{MAP}}}
\newcommand{\be}{\operatorname{Be}} %% probability
\DeclareMathOperator*{\argmin}{arg\,min}
\DeclareMathOperator*{\argmax}{arg\,max}
\DeclareMathOperator\supp{supp}
\usepackage{url}
%%%% Hyperref stuff
\hypersetup{
  colorlinks = true, %Colours links instead of ugly boxes
  urlcolor   = cyan, %Colour for external hyperlinks
  linkcolor  = cyan, %Colour of internal links
  citecolor  = red %Colour of citations
}
%%%% To create without the 'Figure' prefix. Remove if you need'em
\usepackage{caption}
\captionsetup[figure]{labelformat=empty}
%%%%
\author{
\underline{Luiz Max de Carvalho}[lmax.fgv@gmail.com]\linebreak
}
\title{
\Huge Bayesian Statistics
}
\institute{
PhD-level course\\
School of Applied Mathematics (EMAp/FGV), Rio de Janeiro.
}
\date{\today}
\logo{\includegraphics[scale=.15]{logo.jpg}}
\begin{document}
\begin{frame}
\titlepage % Print the title page as the first slide
\end{frame}
\section{Part I: Foundations}
\begin{frame}{Welcome!}
\begin{itemize}
 \item This is a 60-hour, PhD-level course on Bayesian inference.
 \item We have 11 planned weeks. Reading material is posted at~\url{https://github.com/maxbiostat/BayesianStatisticsCourse/}
 \item Assessment will be done via a written exam (70\%) and an assignment ($30\%$);
 \item Tenets:
 \begin{itemize}
  \item Respect the instructor and your classmates;
  \item Read before class;
  \item Engage in the discussion;
  \item Don't be afraid to ask/disagree.
 \end{itemize}
 \item Books are
 \begin{itemize}
  \item  \cite{Robert2007};
  \item \cite{Hoff2009};
  \item  \cite{Schervish1995};
  \item \cite{Bernardo2000}.  
 \end{itemize}
\end{itemize}
\end{frame}

\begin{frame}{Bayes's Theorem}
What do
\begin{equation}
 \label{eq:BT_1}
 \pr(A \mid B) = \frac{\pr(B \mid A)\pr(A)}{\pr(B)},
\end{equation}
and
\begin{equation}
 \label{eq:BT_2}
 \pr(A_i \mid B) = \frac{\pr(B \mid A)\pr(A)}{\sum_{i=1}^n \pr(B \mid A_i)\pr(A_i)},
\end{equation}
and
\begin{equation}
 \label{eq:BT_3}
  p(\theta \mid \boldsymbol{y}) = \frac{l(\boldsymbol{y} \mid \theta)\pi(\theta)}{\int_{\boldsymbol{\Theta}} l(\boldsymbol{y} \mid t)\pi(t) \, dt},
\end{equation}
and
\begin{equation}
 \label{eq:BT_4}
  p(\theta \mid \boldsymbol{y}) = \frac{l(\boldsymbol{y} \mid \theta)\pi(\theta)}{m(\boldsymbol{y})},
\end{equation}
all have in common?
In this course, we will find out how to use Bayes's rule in order to draw statistical inferences in a coherent and mathematically sound way.
\end{frame}
\begin{frame}{Bayesian Statistics is a complete approach}
Our whole paradigm revolves around the posterior:
$$ p(\theta \mid \boldsymbol{x}) \propto l(\theta \mid \boldsymbol{x})\pi(\theta).$$
Within the Bayesian paradigm, you are able to
\begin{itemize}
 \item Perform point and interval inference about unknown quantities;
 \begin{align*}
  \delta(\boldsymbol{x}) &= E_p[\theta] := \int_{\boldsymbol{\Theta}} t p(t \mid \boldsymbol{x} )\,dt,\\
\pr( a \leq \theta \leq b) &= 0.95 = \int_{a}^{b} p(t \mid \boldsymbol{x} )\,dt;
 \end{align*}
\item Compare models:
$$\operatorname{BF}_{12} = \frac{\pr(M_1 \mid \boldsymbol{x})}{\pr(M_2 \mid \boldsymbol{x})} = \frac{\pr(\boldsymbol{x} \mid M_1)\pr(M_1)}{\pr(\boldsymbol{x} \mid M_2)\pr(M_2)};$$
 \item Make predictions: $g(\tilde{x} \mid \boldsymbol{x}) := \int_{\boldsymbol{\Theta}} f(\tilde{x} \mid t)p(t\mid \boldsymbol{x})\,dt$;
 \item Make decisions: $E_p[U(r)]$.
\end{itemize} 
\end{frame}
\begin{frame}{Statistical model: informal definition}
Stuff you say at the bar:
\begin{defn}[Statistical model: informal]
\label{def:statistical_model_informal}
DeGroot, def 7.1.1, pp. 377
A statistical model consists in identifying the random variables of interest (observable and potentially observable), the specification of the joint distribution of these variables and the identification of parameters ($\theta$) that index this joint distribution.
Sometimes it is also convenient to assum that the parameters are themselves random variables, but then one needs to specify a joint distribution for $\theta$ also.
\end{defn} 
\end{frame}
%%%%%%%%%%%%%%%%%%%%%%%%%%%%%%%%%%%
\begin{frame}{Statistical model: formal definition}
Stuff you say in a Lecture:
\begin{defn}[Statistical model: formal]
\label{def:statistical_model_formal}
\href{https://projecteuclid.org/download/pdf_1/euclid.aos/1035844977}{McCullagh, 2002}.
Let $\mathcal{X}$ be an arbitrary sample space, $\Theta$ a non-empty set and $\mathcal{P}(\mathcal{X})$ the set of all probability distributions on $\mathcal{X}$, i.e. $P : \Theta \to [0, \infty)$, $P \in \mathcal{P}$.
 A \underline{parametric} statistical model is a function $P : \Theta \to \mathcal{P}(\mathcal{X})$, that associates each point $\theta \in \Theta$ to a probability distribution $P_\theta$ over $\mathcal{X}$.
\end{defn}
\textbf{Examples}:
\begin{itemize}
 \item Put $\mathcal{X} = \mathbb{R}$ and $\Theta = (-\infty, \infty)\times (0, \infty)$.
 We say $P$ is a \textit{normal} (or \textit{Gaussian}) statistical model\footnote{Note the abuse of notation: striclty speaking, $P_\theta$  is a probability~\textbf{measure} and not a ~\textit{density} as we have presented it here.} if for every $\theta = \{\mu, \sigma^2\} \in \Theta$,
 $$P_{\theta}(x) \equiv \frac{1}{\sqrt{2\pi}\sigma}\exp\left(-\frac{(x-\mu)^2}{2\sigma^2}\right), \: x \in \mathbb{R}.$$
 \item Put $\mathcal{X} = \mathbb{N}\cup \{0\}$ and $\Theta = (0, \infty)$.
 $P$ is a Poisson statistical model if, for $\lambda \in \Theta$,
 $$P_{\lambda}(k) \equiv \frac{e^{-\lambda}\lambda^k}{k!}, \: k = 0, 1, \ldots$$
\end{itemize} 
\end{frame}
% \begin{frame}
% Theorem 
% $$ \int_{\mathcal{X}} f_X(t)\,dt$$ 
%  \begin{theo}[b]
%  a
% \end{theo}
% \end{frame}
% \begin{frame}{Overview}
% \tableofcontents
% \end{frame}

\include{lecture_1}
\subsection{Decision Theory basics}
\begin{frame}{The decision-theoretic foundations of the Bayesian paradigm}
 \begin{defn}[Loss function]
 \label{def:loss_fn} 
 \end{defn}
\end{frame}
%%%%%%%%%%%%%%%%%%%%%%%%%%%%%%%%%%%
\begin{frame}{Utility functions}
Properties:
\end{frame}
%%%%%%%%%%%%%%%%%%%%%%%%%%%%%%%%%%%
\begin{frame}{}
 \begin{theo}[]
   \end{theo}
 See Proposition 4.3 in \cite{Bernardo2000} for a proof outline.
 Here we shall prove the version from~\cite{DeFinetti1931}.
  \begin{idea}[] 
  \end{idea}
\end{frame}
%%%%%%%%%%%%%%%%%%%%%%%%%%%%%%%%%%%
\begin{frame}{Recommended reading}
\begin{itemize}
  \item[\faBook] \cite{Robert2007} Ch. 2. and $^\ast$\cite{Schervish2012} Ch.3;
  \item[\faForward] Next lecture: \cite{Hoff2009} Ch. 2 and $^\ast$\cite{Schervish2012} Ch.1;
 \end{itemize} 
\end{frame}

\subsection{Belief functions and exchangeability}
\begin{frame}{Belief functions}
Let $F, G$ and $H \in \mathcal{S}$ be three (possibly overlapping) statements about the world.
For example, consider the following statements about a person:
\begin{itemize}
 \item [F] = \{votes for a left-wing candidate\} ;
 \item [G] = \{is in the 10\% lower income bracket\} ;
 \item [H] = \{lives in a large city\} ;
\end{itemize}

 \begin{defn}[Belief function]
 \label{def:belief_function} 
 For $A, B \in \mathcal{S}$, a belief function $\be : \mathcal{S} \to \mathbb{R}$ assigns numbers to statements such that $\be(A) < \be(B)$ implies one is more confident in $B$ than in $A$.
 \end{defn}
\end{frame}
%%%%%%%%%%%%%%%%%%%%%%%%%%%%%%%%%%%
\begin{frame}{Belief functions: properties}
It is useful to think of $\be$ as~\textbf{preferences over bets}:
 \begin{itemize}
  \item $\be(F) > \be(G)$ means we would bet on $F$ being true over $G$ being true;
  \item $\be(F\mid H) > \be(G \mid H)$ means that, \textbf{conditional} on knowing $H$ to be true, we would bet on $F$ over $G$;
  \item $\be(F\mid G) > \be(F \mid H)$ means that if we were forced to bet on $F$, we would be prefer doing so if $G$ were true than $H$.
 \end{itemize}
\end{frame}
%%%%%%%%%%%%%%%%%%%%%%%%%%%%%%%%%%%
\begin{frame}{Belief functions: axioms}
 In order for $\be$ to be \textbf{coherent}, it must adhere to a certain set of properties/axioms.
 A self-sufficient collection is:
 \begin{itemize}
  \item [A1]  (boundedness of complete [dis]belief): $$\be(\lnot H \mid H) \leq \be(F \mid H) \leq \be(H \mid H),\, \forall\: F \in \mathcal{S};$$
  \item [A2]  (monotonicity):
  $$\be(F \, \text{or} \, G \mid H) \geq \max \left\{ \be(F \mid H), \be(G \mid H) \right\};$$
  \item [A3] (sequentiality): There exists $f: \mathbb{R}^2 \to \mathbb{R}$ such that
  $$ \be(F\, \text{and} \, G \mid H) = f\left(\be(G\mid H), \be(F \mid G\, \text{and} \, H) \right).$$
 \end{itemize}
\end{frame}
%%%%%%%%%%%%%%%%%%%%%%%%%%%%%%%%%%%
\begin{frame}{Probabilities can be beliefs!}
 \begin{exercise}[Probabilities and beliefs]
  Show that the axioms of belief functions map one-to-one to the axioms of probability:
  \begin{itemize}
   \item[P1.] $0 \leq \pr(E), \forall E \in \mathcal{S}$;
   \item[P2.] $\pr(\mathcal{S}) = 1$;
   \item[P3.] For any countable sequence of disjoint statements $E_1, E_2, \ldots \in \mathcal{S}$ we have
   $$ \pr \left(\bigcup_{i=1}^\infty E_i \right) = \sum_{i=1}^\infty \pr(E_i).$$
  \end{itemize}
 \end{exercise}
Hint: derive the consequences (e.g. monotonicity) of these axioms and compare them with the axioms of belief functions.
\end{frame}
%%%%%%%%%%%%%%%%%%%%%%%%%%%%%%%%%%%
\begin{frame}{Useful probability laws}
\begin{defn}[Partition]
 \label{def:partition}
 If $H = \{H_1, H_2, \ldots, H_k\}$, $H_i \in \mathcal{S}$, such that $H_i \cap H_j = \emptyset$  for all $i \neq j$ and $\bigcup_{k=1}^K = \mathcal{S}$, we say $H$ is a partition of $\mathcal{S}$.
\end{defn}
For any $H \in \mathcal{D}(\mathcal{S})$:
 \begin{itemize}
  \item \textbf{Total probability}: $\sum_{k=1}^K \pr(H_k) = 1$;
  \item \textbf{Marginal probability}: $$\pr(E) = \sum_{k=1}^K = \pr(E \cap H_k) =  \sum_{k=1}^K \pr(E \mid H_k)\pr(H_k),$$
  for all $E \in \mathcal{S}$;
  \item Consequence $\implies$ Bayes's rule:
$$ \pr(H_j \mid E) = \frac{\pr(E \mid H_j)\pr(H_j)}{\sum_{k=1}^K \pr(E \mid H_k)\pr(H_k)}.$$
  \end{itemize}
\end{frame}
%%%%%%%%%%%%%%%%%%%%%%%%%%%%%%%%%%%
\begin{frame}{Independence}
We will now state a central concept in probability theory and Statistics.
 \begin{defn}[ (Conditional) Independence]
  For any $F, G \in \mathcal{S}$, we say $F$ and $G$ are~\textbf{conditionally independent} given $A$ if 
  $$ \pr(F \cap G \mid A) = \pr(F\mid A)\pr(G\mid A).$$  
 \end{defn}
\begin{remark}
 \label{rmk:conditional_indep}
 If $F$ and $G$ are conditionally independent given $A$, then
 $$ \pr(F \mid A \cap G) = \pr(F \mid A).$$
\end{remark}
\begin{proof}
 First, notice that the axioms P1-P3 imply $\pr(F \cap G \mid A) = \pr(G\mid A)\pr(F \mid A \cap G)$.
 Now use conditional independence to write
 \begin{align*}
  \pr(G \mid A) \pr(F \mid A \cap G) &= \pr(F \cap G \mid A) = \pr(F\mid A)\pr(G\mid A),\\
  \pr(G\mid A) \pr(F \mid A \cap G) &= \pr(F\mid A) \pr(G \mid A).
 \end{align*} 
\end{proof} 
\end{frame}
%%%%%%%%%%%%%%%%%%%%%%%%%%%%%%%%%%%
\begin{frame}{Exchangeability} 
\begin{defn}[Exchangeable]
 \label{def:exchangeable}
We say a sequence of random variables $\boldsymbol{Y} = \{ Y_1, Y_2, \ldots, Y_n \}$ are \textbf{exchangeable} if 
$$ \pr(Y_1, Y_2, \ldots Y_n) = \pr(Y_{\xi_1}, Y_{\xi_2}, \ldots Y_{\xi_n}),$$
for all \textbf{permutations} $\boldsymbol{\xi}$ of the labels of $\boldsymbol{Y}$.
\end{defn}
\begin{example}[Uma vez Flamengo... continued]
 Suppose we survey 12 people and record whether they cheer for Flamengo $Y_i = 1$ or not $Y_i = 0$, $i=1, 2,\ldots, 12$.
 What value shoud we assign to :
 \begin{itemize}
  \item $p_1 := \pr(1, 0, 0, 1, 0, 1, 1, 1, 1, 1, 1, 1)$;
  \item $p_2 :=\pr(1, 1, 0, 1, 0, 1, 1, 1, 1, 0, 1, 1)$;
  \item $p_3 := \pr(1, 1, 1, 1, 1, 1, 1, 1, 1, 0, 0, 0)$?
 \end{itemize}
If your answer is $p_1 = p_2 = p_3$ then you are saying the $Y_i$ are (at least partially) exchangeable!
\end{example}
\end{frame}
%%%%%%%%%%%%%%%%%%%%%%%%%%%%%%%%%%%
\begin{frame}{An application of conditional independence}
For $\theta \in (0, 1)$, consider the following sequence of probability statements:
\begin{align*}
\pr(Y_{12} = 1 \mid \theta) &= \theta,\\
\pr(Y_{12} = 1 \mid Y_1, \ldots Y_{11}, \theta) & = \theta,\\
\pr(Y_{11} = 1 \mid Y_1, \ldots Y_{10}, Y_{12}, \theta) &= \theta.
\end{align*}
These imply that the $Y_i$ are conditionally independent and identically distributed (iid), and in particular:
\begin{align*}
 \pr(Y_1 = y_1, \ldots, Y_{12} = y_{12} \mid \theta) &= \prod_{i=1}^{12} \theta^{y_i} (1-\theta)^{1-y_i},\\
 &= \theta^{S} (1-\theta)^{12-S},
\end{align*}
with $S := \sum_{i=1}^{12} y_i$.
Also, under a uniform prior, 
$$ \pr(Y_1, \ldots Y_{12}) = \int_{0}^1 t^{S} (1-t)^{12-S} \pi(t)\,dt = \frac{(S + 1)!(12-S +1)!}{13!} = \binom{13}{S + 1}^{-1}.$$
\end{frame}
%%%%%%%%%%%%%%%%%%%%%%%%%%%%%%%%%%%
\begin{frame}{Relaxing exchangeability (a bit)}
 Sometimes total symmetry can be a burden. 
 We can relax this slightly by introducing the concept of \textbf{partial exchangeability}:
 \begin{defn}[Partially exchangeable]
  \label{def:partially_exchangeable}
  Let $\boldsymbol{X} = \{ X_1, \ldots, X_n\}$ and $\boldsymbol{X} = \{ Y_1, \ldots, Y_m\}$ be two sets of random variables.
  We say $\boldsymbol{X}$ and $\boldsymbol{Y}$ are \textbf{partially} exchangeable if
  $$ \pr\left(X_1, \ldots, X_n ; Y_1, \ldots, Y_m\right) = \pr\left(X_{\xi_1}, \ldots, X_{\xi_n} ; Y_{\sigma_1}, \ldots, Y_{\sigma_m}\right),$$
 \end{defn}
 for any two permutations $\boldsymbol{\xi}$ and $\boldsymbol{\sigma}$ of $1, \ldots, n$ and $1, \ldots, m$, respectively.
 \begin{example}[Uma vez Flamengo...continued]
  To see how exchangeability can be relaxed into partial exchangeability, consider $\boldsymbol{X}$ and $\boldsymbol{Y}$ as observations coming from populations from Rio de Janeiro and Ceará, respectively.
  If the covariate ``state'' were deemed to not matter, then we would have complete exchangeability.
 \end{example}
\end{frame}
%%%%%%%%%%%%%%%%%%%%%%%%%%%%%%%%%%%
\begin{frame}{A statistically useful remark}
 \begin{remark}[Exchangeability from conditional independence]
  \label{rmk:pre_deFinetti}
  Take $\theta \sim \pi(\theta)$, i.e., represent uncertainty about $\theta$ using a probability distribution. 
  If $ \pr(Y_1 = y_1, \ldots, Y_{n} = y_n \mid \theta) = \prod_{i=1}^{n} \pr(Y_i = y_i \mid \theta)$, then $Y_1, \ldots, Y_{n}$ are exchangeable.
 \end{remark}
 \begin{proof}
  Sketch:
  Use
  \begin{itemize}
   \item Marginalisation;
   \item Conditional independence;
   \item Commutativity of products in $\mathbb{R}$;
   \item Definition of exchangeability.
  \end{itemize}
 \end{proof}
\end{frame}
%%%%%%%%%%%%%%%%%%%%%%%%%%%%%%%%%%%
\begin{frame}{A fabulous theorem!}
 \begin{theo}[De Finetti's theorem\footnote{Technically, the theorem stated here is more general than the representation theorem proven by De Finetti in his seminal memoir, which concerned binary variables only.}]
  If $\pr\left(Y_1, \ldots, Y_n\right) = \pr\left(Y_{\xi_1}, \ldots, Y_{\xi_n}\right)$ for all permutations $\boldsymbol{\xi}$ of $1, \ldots, n$, then
  \begin{equation}
   \pr\left(Y_1, \ldots, Y_n\right) = \pr\left(Y_{\xi_1}, \ldots, Y_{\xi_n}\right) = \int_{\boldsymbol{\Theta}} \pr\left(Y_1, \ldots, Y_n \mid t\right) \pi(t)\,dt,
  \end{equation}
for some choice of triplet $\{ \theta,  \pi(\theta), f(y_i \mid \theta) \}$, i.e., a parameter, a prior and a sampling model.
 \end{theo}
 See Proposition 4.3 in \cite{Bernardo2000} for a proof outline.
 Here we shall prove the version from~\cite{DeFinetti1931}.
\end{frame}
%%%%%%%%%%%%%%%%%%%%%%%%%%%%%%%%%%%
\begin{frame}{Consequences}
  This theorem has a few important implications, namely:
 \begin{itemize}
  \item $\pi(\theta)$ represents our beliefs about $\lim_{n\to\infty} \sum_i (Y_i \leq c)/n$ for all $c \in \mathcal{Y}$;
  \item \{ $Y_1, \ldots, Y_n \mid \theta $ are i.i.d \} + \{ $\theta \sim \pi(\theta)$ \} $\iff$ \{ $Y_1, \ldots, Y_n$ are exchangeable for all $n$ \};
  \item If $Y_i \in \{0, 1\}$, we can also claim that:
  \begin{itemize}
   \item If the $Y_i$ are assumed to be independent, then they are distributed Bernoulli conditional on a random quantity $\theta$;
   \item $\theta$ has a prior measure $\Pi \in \mathcal{P}( (0, 1) )$;
   \item By the strong law of large numbers (SLLN), $\theta = \lim_{n \to \infty} (\frac{1}{n}\sum_{i=1}^n Y_i)$, so $\Pi$ can be interpreted as a ``belief about the limiting relative frequency of 1's''.
  \end{itemize}
 \end{itemize}
\end{frame}
%%%%%%%%%%%%%%%%%%%%%%%%%%%%%%%%%%%
\begin{frame}{The soul of Statistics}
 As the exchangeability results above clearly demonstrate, being able to use conditional independence is a handy tool.
 More specifically, knowing on what to condition so as to make things exchangeable is key to statistical analysis.
 \begin{idea}[Conditioning is the soul of Statistics\footnote{This idea is due to Joe Blitzstein, who did his PhD under no other than the great Persi Diaconis.}] 
 \label{idea:conditioning_soul}
 Knowing on what to condition can be the difference between an unsolvable problem and a trivial one.
 When confronted with a statistical problem, always ask yourself ``What do I know for sure?'' and then ``How can I create a conditional structure to include this information?''.
 \end{idea}
\end{frame}
%%%%%%%%%%%%%%%%%%%%%%%%%%%%%%%%%%%
\begin{frame}{Recommended reading}
\begin{itemize}
  \item[\faBook] \cite{Hoff2009} Ch. 2 and $^\ast$\cite{Schervish2012} Ch.1;
 \item $^\ast$Paper: \cite{Diaconis1980} explains why if $n$ samples are taken from an exchangeable population of size $N \gg n$ without replacement, then the sample $Y_1, \ldots Y_n$ can be modelled as approximately exchangeable;
 \item[\faForward] Next lecture: \cite{Robert2007} Ch. 3.
 \end{itemize} 
\end{frame}

\include{lecture_4}
\section*{Bayesian point estimation}
\begin{frame}{The maximum~\textit{a posteriori} (MAP) estimator}
\begin{defn}[Maximum~\textit{a posteriori}]
The posterior mode or maximum~\textit{a posteriori} (MAP) estimator of a parameter $\theta$ is given by
\begin{equation}
 \label{eq:MAP}
 \delta_{\pi}^{\text{MAP}}(x) := \argmax_{\theta \in \boldsymbol{\Theta}} p(\theta \mid x).
\end{equation}
\end{defn}
\begin{example}[MAP for the binomial case]
 Suppose $x \sim \operatorname{Binomial}(n, p)$.
 Now consider the following three priors for $p$: 
 \begin{itemize}
  \item $\pi_0(p) = \frac{\sqrt{p(1-p)}}{B(1/2, 1/2)} $ [Jeffreys];
  \item $\pi_1(p) = 1$ [Beta(1,1)/Uniform];
  \item $\pi_2(p) = \left(p(1-p)\right)^{-1}$ [\cite{Haldane1932}].  
 \end{itemize}
These lead to 
\begin{itemize}
 \item $\delta_0^{\text{MAP}}(x) = \max \{(x-1/2)/(n-1), 0 \}$;
 \item $\delta_1^{\text{MAP}}(x) = x/n$;
 \item $\delta_2^{\text{MAP}}(x) = \max \{ (x-1)/(n-2), 0 \}$.
\end{itemize}
\end{example}
\end{frame}
%%%%%%%%%%%%%%%%%%%%%%%%%%%%%%%%%%%
\begin{frame}{A word of caution}
 We end this discussion with the following warning:
\begin{idea}[Marginalise, not maximise]
\label{idea:always_marginalise}~
 Bayesian approaches to estimation and prediction \underline{usually} focus on \textit{marginalisation} rather than \textit{optmisation}.
 This is because, following the Likelihood Principle, all of the information available about the unknowns is contained in the posterior distribution, and thus all inferences must be made using this probability measure, usually by finding suitable expectations of functionals of interest.
\end{idea}
In particular, for higher dimensions, \textbf{concentration of measure}\footnote{See these excellent notes by Terence Tao:\url{https://terrytao.wordpress.com/2010/01/03/254a-notes-1-concentration-of-measure/} .} ensures that the posterior mode has less and less relevance as a summary, at least so far as the barycentre of the distribution is concerned.
\end{frame}
%%%%%%%%%%%%%%%%%%%%%%%%%%%%%%%%%%%
\begin{frame}{Precision of Bayes estimators}
 A central quantity in the evaluation of Bayesian estimators is
 \begin{equation}
  \label{eq:bayes_risk}
  E_p\left[\left(\delta_\pi - h(\theta)\right)^2\right] = E_{\pi}\left[\left(\delta_\pi - h(\theta)\right)^2 \mid x \right]
 \end{equation}
for measurable $h$.
\begin{example}[Bayes versus frequentist risk]
 Take $x \sim \operatorname{Binomial}(n, \theta)$ with $n$ known and place a Jeffreys's prior on $\theta$.
 Consider the MLE: $\delta_1(x) = x/n$.
 It can be shown that:
 $$ E_\pi\left[\left(\delta_1 - \theta \right)^2 \mid x \right] = \left(\frac{x - n/2}{n(n+1)}\right)^2 +  \frac{(x + 1/2)(n - x + 1/2)}{(n + 1)^2(n+2)}.$$
 Moreover,
 $$\max_{\theta \in (0, 1)}  E_\pi\left[\left(\delta_1 - \theta\right)^2 \mid x \right] = [4(n+2)]^{-1},$$
 and
 $$ \max_{\theta \in (0, 1)} E_\theta \left[\left(\delta_1 - \theta\right)^2\right] = [4n]^{-1}.$$ 
\end{example}
\end{frame}
%%%%%%%%%%%%%%%%%%%%%%%%%%%%%%%%%%%
\begin{frame}{A brief aside about prediction}
 Prediction is an important inferential task and is somewhat related to the previous discussion on precision.
 Consider predicting a quantity $z$ \textbf{conditional} on data $x$.
 For that  we need $g(z \mid x, \theta)$, $f(x\mid \theta)$ and $\pi(\theta)$.
 Then,
 \begin{equation}
  \label{eq:general_predictive}
  g_\pi(z\mid x) = \int_{\boldsymbol{\Theta}} g(z\mid x, t) p(t \mid x)\,dt
 \end{equation}
encodes all of the information brought by the posterior about $z$.
A special case  is i.i.d prediction:
\begin{equation}
 \label{eq:posterior_predictive_data}
 g(\tilde{x} \mid x) = \int_{\boldsymbol{\Theta}} f(\tilde{x} \mid t) p(t \mid x)\,dt
\end{equation}
is the posterior predictive of the new data $\tilde{x}$.
\begin{idea}[Calibrated priors for prediction]
\label{idea:prediction_calibrated_priors}
The prior, $\pi$, can be constructed so as to minimise error in a prediction task. 
\end{idea}
\end{frame}
%%%%%%%%%%%%%%%%%%%%%%%%%%%%%%%%%%%
\begin{frame}{A neat trick}
Computing expectations all the time means we have to become familiar with a few tricks to facilitate obtaining approximate answers.
\begin{example}[Mixture representation of the Student-t]
Take $x \sim \operatorname{Normal}_p(\theta, \boldsymbol{I}_p)$ and put $\theta \sim \operatorname{Student-t}_p(\alpha, 0, \tau^2\boldsymbol{I}_p)$.
Then $p(\theta \mid x)$ does not have a closed-form normalising constant and computing the Bayes estimator under quadratic loss is a chore.
However, we can use the representation
\begin{align*}
 \theta \mid z &\sim \operatorname{Normal}_p(0, \tau^2z\boldsymbol{I}_p),\\
 z & \sim \operatorname{InverseGamma}(\alpha/2, \alpha/2),
\end{align*}
to get 
$$ \theta \mid x, z \sim  \operatorname{Normal}_p\left(\frac{x}{ 1+ \tau^2z}, \frac{\tau^2z}{ 1+ \tau^2z}\boldsymbol{I}_p\right) $$ 
Thus, the Bayes estimator $\delta_\pi(x) = \int_0^\infty E_\pi[\theta \mid x, z]p(z \mid x)\,dz$ can be computed with a single integral for any dimension $p$. 
\end{example} 
\end{frame}
%%%%%%%%%%%%%%%%%%%%%%%%%%%%%%%%%%%
\begin{frame}{Conjugacy is handy!\footnote{Taken from~\cite{Robert2007}.}}
\begin{center}
 \includegraphics[scale=0.5]{figures/conjugate_table_expectations.pdf}
\end{center}
\end{frame}
%%%%%%%%%%%%%%%%%%%%%%%%%%%%%%%%%%%
\begin{frame}{A worked example}
We will stretch our Bayesian muscles with the next problem. 
\begin{exercise}[Inference for the rate of a Gamma]
\label{exercise:rate_gamma_different_losses}
 Let $x \sim \operatorname{Gamma}(\nu, \theta)$ with $\nu >0$ known.
 A natural choice of prior is $\theta \sim \operatorname{Gamma}(\alpha, \beta)$.
 Find the Bayes estimator under 
 $$ L_1(\delta, \theta) = \left(\delta - \frac{1}{\theta}\right)^2, $$
 and the scale-invariant loss
 $$ L_2(\delta, \theta) = \theta^2 \left(\delta - \frac{1}{\theta}\right)^2$$
 \textit{Hint:} If $X \sim \operatorname{Gamma}(\alpha, \beta)$, $Y = 1/X \sim \operatorname{InverseGamma}(\alpha, \beta)$ and $E[Y^k] = \frac{\beta^k}{(\alpha-1)\cdots(\alpha-k)}$
\end{exercise}
\end{frame}
%%%%%%%%%%%%%%%%%%%%%%%%%%%%%%%%%%%
\begin{frame}{A quick note on quadratic loss}
 Exercise~\ref{exercise:rate_gamma_different_losses} is a special case of the general situation where 
 $$ L(\delta, \theta) = w(\theta) ||\delta-\theta||_{\boldsymbol{G}}^2, $$
 for $\boldsymbol{G}$ a $p \times p$ non-negative symmetric matrix.
 In this case, we get
 $$ \delta_\pi = \frac{E_p[w(\theta)\theta]}{E_p[w(\theta)]}. $$
 Please \textbf{note} that there is no universal justification for quadratic loss other than (sometimes leading to increased) mathematical tractability.
\end{frame}
%%%%%%%%%%%%%%%%%%%%%%%%%%%%%%%%%%%
\begin{frame}{Loss estimation}
 Since the loss function, $L(\delta(x), \theta)$ is usually measurable w.r.t the posterior, it can be estimated much the same way as other functionals.
 In particular, if you are feeling particularly eclectic, you can always constructed $\pi$ such that
 $$ E\left[E_p[L(\delta_\pi(x), \theta]\right] \geq R(\delta_\pi(x), \theta),  \theta \in \boldsymbol{\Theta}, $$
 i.e. that the estimated loss never underestimates the error resulting from the use of $\delta_\pi$, at least in the long run.
 This is called \textbf{frequentist validity}.
 \end{frame}
%%%%%%%%%%%%%%%%%%%%%%%%%%%%%%%%%%%
\begin{frame}{A nice little problem by Neyman}
The following problem is described by Jeffreys as originating with Jerzy Neyman\footnote{Jerzy Neyman (1894-1981) was a Polish-American statistician, known for with work with Egon Pearson (1895-1980) on the foundations of the null hypothesis significance testing (NHST) framework.}.
\begin{exercise}[The tramcar problem]
 \label{exercise:tramcar}
 A person travelling in a a foreign country has to change trains at a junction, and goes into the town, the existence of which they have only just heard.
 They have no idea of its size.
 The first thing they see is a tramcar numbered $100$.
 Assuming tramcars are numbered consecutively from $1$ onwards, what could one \textit{infer} about the number $N$ of tramcars in this town?
\end{exercise}
\end{frame}

%%%%%%%%%%%%%%%%%%%%%%%%%%%%%%%%%%%
\begin{frame}{Recommended reading}
\begin{itemize}
  \item[\faBook] \cite{Robert2007}, Ch4.
%  \item 
 \item[\faForward] Next lecture: \cite{Robert2007} Ch. 5.
 \end{itemize} 
\end{frame}

\include{lecture_6}
%%%%%%%%%%%%%%%%%%%%%%%%%%%%%%%%%%%
\section*{Bayesian model choice}
\begin{frame}{Bayesian model selection: testing all over again}
 Model choice (or selection) is a \textbf{major} topic within any school of inference: it is how scientists make decisions about competing theories/hypotheses in light of data.
 One can associate a set of models $\boldsymbol{\mathcal{M}} = \{ \mathcal{M}_1, \ldots \mathcal{M}_n \}$ with a set of indices $I$ such that $\mu \in I$ we want to estimate the posterior distribution of the indicator function $\mathbb{I}_{\boldsymbol{\Theta}_\mu}(\theta)$.
 
 Recall that estimating indicator functions over $\boldsymbol{\Theta}$ was the fundamental mechanic of Bayesian testing.
 In the setting of Bayesian model selection (BMS), we have something of the form
 \begin{equation*}
  \mathcal{M}_i : x \sim f_i(x \mid \theta_i), \theta_i \in \boldsymbol{\Theta}_i, i \in I.
 \end{equation*}
\end{frame}
%%%%%%%%%%%%%%%%%%%%%%%%%%%%%%%%%%%
\begin{frame}{M-completeness}
A key step in model selection is to identify in which regime the analyst finds themselves in.
\begin{defn}[M-open, M-closed, M-complete]
 \label{def:m-open}
 Model selection can be categorised in three settings:
 \begin{itemize}
  \item \textbf{M-closed}: a situation where the true data-generating model is one of $\mathcal{M}_i \in \boldsymbol{\mathcal{M}}$, even though  it is most often  unknown to the  analyst;
  \item \textbf{M-complete}: a situation where the true model exists and is out of the model set $\boldsymbol{\mathcal{M}}$.
  We nevertheless want to select one of the models in the set due to computational or mathematical tractability reasons.
  \item \textbf{M-open}: a situation in which we know the true data-generating model is not in $\boldsymbol{\mathcal{M}}$ and we have no idea what it looks like.
 \end{itemize}
\end{defn}
See~\cite{Bernardo2000} and \cite{Yao2018}.
\end{frame}
%%%%%%%%%%%%%%%%%%%%%%%%%%%%%%%%%%%
\begin{frame}{BMS: example I}
Suppose one has $x \in \mathbb{N}\cup \{0\}$, which measures, say, the number of eggs Balerion The Black Dread has laid in five consecutive breeding seasons.
One can conjure up
\begin{equation*}
 \mathcal{M}_1 : x \sim \operatorname{Poisson}(\lambda), \lambda > 0,
\end{equation*}
or, if feeling fancy, 
\begin{equation*}
 \mathcal{M}_2 : x \sim \operatorname{Negative-binomial}(\lambda, \phi), \lambda, \phi > 0.
\end{equation*}
Notice that, under $\mathcal{M}_2$, $E[X] = \lambda$ and $\vr(X) = \lambda ( 1 + \lambda/\phi)$.
What happens as $\phi \to \infty$?
\end{frame}
%%%%%%%%%%%%%%%%%%%%%%%%%%%%%%%%%%%
\begin{frame}{BMS: example II}
Take the famous Galaxy data set:
 \begin{center}
 \includegraphics[scale=0.3]{figures/galaxies.pdf}
\end{center}
A now classical model is a Gaussian mixture:
\begin{equation*}
 \mathcal{M}_i : v_j \sim \sum_{l=1}^i p_{il} \cdot \operatorname{Normal}(v_j; \mu_{li},\sigma^2_{li}). 
\end{equation*}
\end{frame}
%%%%%%%%%%%%%%%%%%%%%%%%%%%%%%%%%%%
\begin{frame}{BMS: example III}
Consider the data:
 \begin{center}
 \includegraphics[scale=0.3]{figures/oranges.pdf}
\end{center}
Amongst the models we can consider, 
\begin{align*}
\mathcal{M}_1 :& y_{it} \sim \operatorname{Normal}(\beta_{10} + b_{1i}, \sigma_1^2), \\
\mathcal{M}_2 :& y_{it} \sim \operatorname{Normal}(\beta_{20} + \beta_{21}T_t + b_{2i}, \sigma_2^2) , \\
\mathcal{M}_3 :& y_{it} \sim \operatorname{Normal}\left(\frac{\beta_{30}}{1 + \beta_{31}\exp\left(\beta_{32} T_t\right)}, \sigma_3^2\right), \\
\mathcal{M}_4 :& y_{it} \sim \operatorname{Normal}\left(\frac{\beta_{40} + b_{4i}}{1 + \beta_{41}\exp\left(\beta_{42} T_t\right)}, \sigma_4^2\right).
\end{align*}
\end{frame}
%%%%%%%%%%%%%%%%%%%%%%%%%%%%%%%%%%%
\begin{frame}{Step 0: priors}
First, let us look at a convenient representation of model space:
\begin{equation*}
 \boldsymbol{\Theta} = \bigcup_{i \in I} \{i\} \times \boldsymbol{\Theta}_i.
\end{equation*}
Now, to each $\mathcal{M}_i$, we associate a prior $\pi_i(\theta_i)$  on each subspace and, by Bayes' theorem we get
\begin{align*}
 \pr(\mathcal{M}_i \mid x) & = \pr( \mu = i \mid x), \\ 
 &= \frac{w_i \int_{\boldsymbol{\Theta}_i} f_i(x\mid t_i)\pi_i(t_i)\,dt_i}{\sum_{j} w_j \int_{\boldsymbol{\Theta}_j} f_j(x\mid t_j)\pi_j(t_j)\,dt_j },
\end{align*}
where the $w_i$ are the \textbf{prior probabilities} for each model.
\end{frame}
%%%%%%%%%%%%%%%%%%%%%%%%%%%%%%%%%%%
\begin{frame}{An intuitive predictive}
A nice consequence of the formulation we just saw is that the predictive distribution looks quite intuitive:
\begin{align}
\nonumber
 p(\tilde{x} \mid \boldsymbol{x}) &= \sum_{j} w_j \frac{1}{m_j(\boldsymbol{x})}\int_{\boldsymbol{\Theta}_j} f_j(\tilde{x} \mid t_j) f_j(\boldsymbol{x}\mid t_j)\pi_j(t_j)\,dt_j,\\
 \label{eq:predictive_1}
 &= \sum_{j} \pr(\mathcal{M}_j \mid \boldsymbol{x}) p_j(\tilde{x} \mid \boldsymbol{x}).
\end{align}
\end{frame}
%%%%%%%%%%%%%%%%%%%%%%%%%%%%%%%%%%%
\begin{frame}{Hello, my old friend}
Here, Bayes factors also play a central role:
\begin{align*}
 \operatorname{BF}_{12} &= \frac{\pr(\mathcal{M}_1 \mid x)}{\pr(\mathcal{M}_2 \mid x)}\bigg/\frac{\pr(\mathcal{M}_1)}{\pr(\mathcal{M}_2)},\\
  &= \frac{w_1^\prime \cdot w_2}{w_2^\prime \cdot w_1},
\end{align*}
with $w_i^\prime := \pr(\mathcal{M}_1 \mid x)$.
\end{frame}
%%%%%%%%%%%%%%%%%%%%%%%%%%%%%%%%%%%
\begin{frame}{Model averaging}
What if we simply \textbf{refuse} to select one model?
We can write
\begin{align}
 \nonumber
 p(\tilde{x} \mid \boldsymbol{x})  &= \int_{\boldsymbol{\Theta}} f(\tilde{x} \mid t) f(\boldsymbol{x}\mid t)\pi(t)\,dt,\\
 \nonumber
 &= \sum_{j} \int_{\boldsymbol{\Theta}_j} f_j(\tilde{x} \mid t_j) g(j, t_j \mid \boldsymbol{x})\,dt_j,\\
 \nonumber
 &= \sum_j p (\mathcal{M}_j \mid \boldsymbol{x}) \int_{\boldsymbol{\Theta}_j} f_j(\tilde{x} \mid t_j) p(t_j \mid \boldsymbol{x})\,dt_j,\\
 \label{eq:predictive_2}
  &= \sum_j w_j^\prime \int_{\boldsymbol{\Theta}_j} f_j(\tilde{x} \mid t_j) p(t_j \mid \boldsymbol{x})\,dt_j.
\end{align}
which is another version of the expression in (\ref{eq:predictive_1}).
\end{frame}
%%%%%%%%%%%%%%%%%%%%%%%%%%%%%%%%%%%
\begin{frame}{Model checking}
Modern Bayesian inference not only allows for, but actively encourages model interrogation and checking.
\begin{itemize}
 \item The central idea of \textbf{Leave-one-out cross-validation (LOO)} is to estimate the \textit{expected log pointwise predictive density for a new dataset}, elpd:
 \begin{equation*}
  \operatorname{elpd} = \sum_{i=1}^n \int m(\tilde{x}_i)\log p(\tilde{x}_i \mid \boldsymbol{x})\,d\tilde{x}_i.
 \end{equation*}
 See \cite{Vehtari2017}.
 \item With \textbf{Posterior predictive checks (PPCs)} we wish to compare  functions of the observed data, $f(\boldsymbol{x})$ with functions of the predictive distribution, $f(\boldsymbol{\tilde{x}})$.
  \begin{center}
 \includegraphics[scale=0.5]{figures/PPC.jpg}
\end{center}
 See \cite{Berkhof2000} and~\cite{Gabry2019}.
\end{itemize}
\end{frame}
%%%%%%%%%%%%%%%%%%%%%%%%%%%%%%%%%%%
\begin{frame}{Recommended reading}
\begin{itemize}
  \item[\faBook] \cite{Robert2007}, Ch. 7.
%  \item 
 \item[\faForward] Next lecture: \cite{Schervish1995} Ch. 7.4.
 \end{itemize} 
\end{frame}

\section*{Bayesian asymptotics}
\begin{frame}{Asymptotics}
A major part of a statistical approach is understanding what happens in the limit of many many observations.
Consider the joint conditional density of the data, $f_n(\boldsymbol{x} \mid \theta)$ and a prior $\pi(\theta)$.
What happens to $p_n(\theta \mid \boldsymbol{x}) = f_n(\boldsymbol{x} \mid \theta)\pi(\theta)/m_n(\boldsymbol{x})$ as $n \to \infty$ ?
\begin{idea}[Asymptotics is about understanding]
 Infinity is a big ``number''.
 Considering what happens  as $n \to \infty$ is less a statement about a real world situation than about the structure and regularity of a model.
 Doing asymptotics is about understanding what makes a model tick rather than getting useful results for a regime seldom achieved in practice.
\end{idea}
Another important aspect to consider is the \textbf{rate} at which things converge asymptotically.
Studying rates provides complementary information about the structure of the model and gives hints as to the accuracy of asymptotic approximations.
 \end{frame}
%%%%%%%%%%%%%%%%%%%%%%%%%%%%%%%%%%%
\begin{frame}{Bayesian asymptotics I: consistency}
\begin{theo}[The posterior concentrates around the ``true'' value]
Let $(S, \mathcal{A}, \mu)$ be a probability space and let $(\Omega, \tau)$ be a finite-dimensional parameter space equipped with a Borel $\sigma$-field.
Suppose there exist measurable $h_n: \mathcal{X}^n \to \Omega$ such that $h_n(\boldsymbol{X}_{n})$ converges in probability to $\Theta$.
Writing $\mu_{\boldsymbol{\Theta} \mid \boldsymbol{X}_{n}}(\cdot \mid \boldsymbol{x}_{n})$ for the posterior measure, we have
\begin{equation*}
 \lim_{n \to \infty} \mu_{\boldsymbol{\Theta} \mid \boldsymbol{X}_{n}}(A \mid \boldsymbol{X}_{n}) = I_A(\Theta), \: \mu-\textrm{a.s.}
\end{equation*}
\end{theo}
\textbf{Please} see Theorem 7.78 in \cite{Schervish1995} (pg 429) for all of the \textit{many} details.

\textbf{Discussion:} what we are essentially saying here is that if there exists a consistent (sequence of) estimator(s) for $\theta$ -- understood here as a random variable --, then the posterior will concentrate around the true generating distribution of the parameter asymptotically.
\end{frame}
%%%%%%%%%%%%%%%%%%%%%%%%%%%%%%%%%%%
\begin{frame}{Remember Cromwell's law?}
Here is another neat little theorem with a cumbersome proof.
\begin{theo}[A ``nice'' prior ensures posterior consistency]
Define $\operatorname{KL}(\theta, \theta^\prime)$ as the Kullback-Leibler divergence between $P_{\theta}$ and $P_{\theta^\prime}$.
Let $\theta_0$ be the true data-generating parameter and define $C_\epsilon = \{\theta : \operatorname{KL}(\theta_0, \theta) < \epsilon\}$, $\epsilon > 0$.
 Let $\Pi$ be a prior measure such that $\Pi(C_\epsilon) > 0$ for every $\epsilon > 0$.
 Take $N_0$ open such that $C_\epsilon \subset N_0$.
 Then 
 \begin{equation*}
  \lim_{n \to \infty} \mu_{\boldsymbol{\Theta} \mid \boldsymbol{X}_{n}}(N_0 \mid \boldsymbol{X}_{n}) = 1, \: P_{\theta_0}-\textrm{a.s.}
 \end{equation*}
\end{theo}
Again, \textbf{please} see Theorem 7.80 in \cite{Schervish1995} (pg 430) for the details.
\end{frame}
%%%%%%%%%%%%%%%%%%%%%%%%%%%%%%%%%%%
\begin{frame}[allowframebreaks]{Interlude: regularity conditions}
Before we proceed, we will need to make things nice.
Consider the following regularity conditions
\begin{itemize}
 \item[1] The parameter space is $\boldsymbol{\Theta} \subset \mathbb{R}^d$ for some finite $d$;
 \item[2] We have $\theta_0$ an an interior point of $\boldsymbol{\Theta}$;
 \item[3] The prior distribution has a density w.r.t. Lebesgue which is positive and continuous at $\theta_0$;
 \item[4] There exists $N_0 \subseteq \boldsymbol{\Theta}$ with $\theta_0 \in N_0$ such that the log-likelihood, $l_n(\theta)$, is twice-differentiable with respect to all coordinates of $\theta$, $P_{\theta}$-a.s.
 \item[5] The largest eigenvalue of the inverse observed Fisher information, $\Sigma_n$, vanishes in probability.
 \item[6] The MLE is consistent;
 \item[7] The Fisher information is a smooth function of $\theta$.
\end{itemize}

\end{frame}
%%%%%%%%%%%%%%%%%%%%%%%%%%%%%%%%%%%
\begin{frame}{Bayesian asymptotics II: asymptotic normality}
We can now state a nice result which characterises the asymptotic form of the posterior.
\begin{theo}[Bernstein von-Mises\footnote{Named after Austrian mathematician Richard Edler von Mises (1883--1953) and Russian mathematician Sergei Natanovich Bernstein (1880--1968).}]
Under the regularity conditions we have discussed, take $\hat{\theta}$ to be the MLE.
Put $\boldsymbol{\Psi}_n = \left(\Sigma_n\right)^{-1/2}(\theta- \hat{\theta})$.
Then the posterior distribution of $\boldsymbol{\Psi}_n$ conditional on $\boldsymbol{X}$ converges in probability \textbf{uniformly} on compact sets to the multivariate normal distribution $\operatorname{Normal}_d\left(\boldsymbol{0}, \boldsymbol{I}_d\right)$ with density $\phi_d$.
More precisely,
\begin{equation*}
  \lim_{n \to \infty} P_{\theta_0}\left(\sup_{\psi \in B} \bigg\rvert f_{\boldsymbol{\Psi}_n \mid \boldsymbol{X}}(\psi) - \phi_d(\psi)  \bigg\lvert > \epsilon \right) = 0,
\end{equation*}
for all $B \subset \mathbb{R}^d$ compact and $\epsilon > 0$.
\end{theo}
See Theorem 7.89 in \cite{Schervish1995} (page 437).
\end{frame}
%%%%%%%%%%%%%%%%%%%%%%%%%%%%%%%%%%%
\begin{frame}{Dabbling with normal approximations}
\begin{exercise}[Cauchy location posterior]
 Take $X_i \sim \operatorname{Cauchy}(\theta, 1)$, $i = 1, 2,\ldots, 10$.
 In particular, suppose $\boldsymbol{x} = \{-5, -3, 0, 2, 4, 5, 7, 9, 11, 14\}$.
 \begin{itemize}
  \item[i)] Compute the MLE and $l^{\prime\prime}$;
  \item[ii)] Deduce the parameters of the normal approximation to $p(\theta \mid \boldsymbol{x})$;
  \item[iii)] Use an MCMC\footnote{The instructor can assist with this step.} routine to sample from $p(\theta \mid \boldsymbol{x})$, obtain a posterior approximation to its density and compare it to the normal approximation;
  \item[iv)] Simulate data sets of sizes $n=20, 50, 100, 500, 1000$ and $10, 000$ and repeat iii.
  \item[v)] See if you can reduce/increase the discrepancy between the posterior and its approximation by fiddling with the prior (without breaking the regularity assumptions!).
 \end{itemize}
\end{exercise}
See example 7.104 in \cite{Schervish1995} (page 444).
 \end{frame}
%%%%%%%%%%%%%%%%%%%%%%%%%%%%%%%%%%%
\begin{frame}{Recommended reading}
\begin{itemize}
  \item[\faBook] \cite{Schervish1995} Ch. 7.4.
%  \item 
 \item[\faForward] Next lecture: \cite{Raftery1988} and~\cite{Gelman2002}.
 \end{itemize} 
\end{frame}

\section*{Bayesian rules}
\begin{frame}{Why be Bayesian I: probabilistic representation}

We already reduce our uncertainty about phenomena to probability distributions for sampling distributions (likelihoods).

\begin{idea}[Probabilisation of uncertainty]
 \label{id:prob_uncertainty}
 Our statistical models are \textit{interpretations} of reality, rather than \textit{explanations} of it.
 Moreover,
 \begin{quote}
 `` ...the representation of unknown phenomena by a probabilistic model,  at the observational level as well as at the parameter level, does not need to correspond effectively—or physically—to a generation from a probability distribution, nor does it compel us to enter a supradeterministic scheme, fundamentally because of the nonrepeatability of most experiments.''
 \end{quote}
 \cite{Robert2007}, pg 508.
\end{idea}
 \end{frame}
%%%%%%%%%%%%%%%%%%%%%%%%%%%%%%%%%%
\begin{frame}{Why be Bayesian II: conditioning on OBSERVED data}
  Remember Idea~\ref{id:soul}: conditioning is the soul of (Bayesian) Statistics.
  \begin{idea}[Conditioning on what is actually observed]
   \label{id:obs_data}
   A quantitative analysis about the parameter(s) $\theta$ conditioning \textit{only} on the observed data, $x$ unavoidably requires a distribution over $\theta$.
   To this end, the \textbf{only} coherent way to achieve this goal starting from a distribution $\pi(\theta)$ is to use Bayes's theorem.
  \end{idea}
  
  Frequentist arguments are, necessarily, about procedures that behave well under a given data-generating process and thus forcibly make reference to unobserved data sets that could, in theory, have been observed.
 \end{frame}
%%%%%%%%%%%%%%%%%%%%%%%%%%%%%%%%%%
\begin{frame}{Why be Bayesian III: priors as inferential tools}

A refreshing break from the strictly subjectivist view of Bayesianism can be had if we think about inference functionally. 

 \begin{idea}[The prior as a regularisation tool]
 \label{id:prior_tool}
  If one adopts a mechanistic view of Bayesian inference, the prior can be seen as an additional regularisation or penalty term that enforces certain model behaviours, such as sparsity or parsimony.
  A good prior both \textit{summarises} substantitve knowledge about the process and rules out unlikely model configurations.
 \end{idea}

 In other words, sometimes it pays to use the prior to control what the model \textit{does}, rather than which specific values the parameter takes.
 
 \end{frame}
%%%%%%%%%%%%%%%%%%%%%%%%%%%%%%%%%%
\begin{frame}{Why be Bayesian IV: embracing subjectivity}
The common notion of ``objectivity'' is ruse.
There is no such thing as a truly objective analysis, and taking objectivity as premise might hinder our ability to focus on actual discovery and explanation~\citep{Hennig2017}.
\begin{idea}[The subjective basis of knowledge]
\label{id:subjective}
 Knowledge arises from a confrontation between \textit{a prioris} and experiments (data).
 Let us hear what Poincaré\footnote{Jules Henri Poincaré (1854--1912) was a French mathematician and the quote is from \textit{La Science and l'Hypóthese} (1902).} had to say:
 \begin{quote}
  ``It is often stated that one should experiment without preconceived ideas.
  This is simply impossible; not only would it make every experiment sterile, but even if we were ready to do so, we could not implement this principle. 
  Everyone stands by [their] own conception of the world, which [they] cannot get rid of so easily.''
 \end{quote}
 \end{idea}
\end{frame}
%%%%%%%%%%%%%%%%%%%%%%%%%%%%%%%%%%
\begin{frame}{Why be Bayesian V: principled inference}
As we saw in the first lectures of this course, the Bayesian approach is coherent with a few very compelling principles, namely Sufficiency, Conditionality and the Likelihood principle.
\begin{idea}[Bayesian inference follows from strong principles]
Starting from a few desiderata, namely conditioning on the \textbf{observed} data, independence of stopping criteria and respecting the sufficiency, conditionality and  likelihood principles, one arrives at a single approach: Bayesian inference using proper priors.
 \label{id:principled_inference}
\end{idea}
 \end{frame}
%%%%%%%%%%%%%%%%%%%%%%%%%%%%%%%%%%
\begin{frame}{Why be Bayesian VI: universal inference}
Bayesian Statistics provides an universal procedure for drawing inference about probabilistic models, something Frequentists can only dream of.
\begin{idea}[Bayesian inference is universal]
Starting from a sampling model, a (proper) prior and a loss (or utility) function, the Bayesian analyst can always derive an estimator.
Moreover, and importantly, many optimal frequentist estimators can be recovered from Bayesian estimators or limits of Bayesian estimators.
Paradoxically, this means that one can be a staunch advocate of Frequentism and still employ Bayesian methods (see, e.g. least favourable priors). 
 \label{id:universal}
\end{idea}
\end{frame}
%%%%%%%%%%%%%%%%%%%%%%%%%%%%%%%%%%
\begin{frame}{How to be Bayesian I: clarity \& openness}
Being subjective does not mean ``anything goes''.
As a scientist, you are still bound by the laws of logic and reason.

\begin{idea}[State your prior elicitation clearly and openly]
 As we have seen, prior information does not always translate exactly into one unique prior choice.
 In other words, the same prior information can be represented adequately by two or more probability distributions.
 Make sure your exposition \textbf{clearly} separates which features of the prior come from substantitve domain expertise and which ones are arbitrary constraints imposed by a particular choice of parametric family, for instance.
 An effort must be made to state all modelling choices \textbf{openly}.
 Openly stating limitations is not a bug, it is a feature.
 \end{idea}
\end{frame}
%%%%%%%%%%%%%%%%%%%%%%%%%%%%%%%%%%
\begin{frame}{How to be Bayesian II: ``noninformativeness'' requires care}
Mathematically, Bayesian Statistics is all-encompassing (see Idea~\ref{id:universal}).
One must be careful\footnote{Personally, I'm not opposed to reference priors and the like, and gladly employ them in my own research work, but I do think one needs to know very well what one is doing in order to employ them properly.} when employing so-called ``objective'' Bayesian methods.
\begin{idea}[Beware of objective priors]
 \label{id:careful}
 In a functional sense, non-informative priors are a welcome addition to Bayesian Statistics because they provide~\textit{closure}, and confer its universality.
 On the other hand, reference priors and the like cannot be justified as summarising prior information.
 From a technical standpoint, many noninformative priors are also improper and thus impose the need to check propriety of the resulting posterior distribution.
\end{idea}
 \end{frame}
%%%%%%%%%%%%%%%%%%%%%%%%%%%%%%%%%%
\begin{frame}{A word of caution}

A strong defence of the Bayesian paradigm should not cloud our view of the bigger picture.
Statistics is the grammar of Science; whatever grammatical tradition you choose, be sure to employ it properly.
\begin{idea}[Do not become a zealot!]
 \label{id:not_zealot}
 Statistics is about learning from data and making decisions under uncertainty.
 The key to a good statistical analysis is not which ideology underpins it, but how helpful it is at answering the scientific questions at hand.
 Ideally, you should know both\footnote{Here we are pretending for a second that there are only two schools of thought in Statistics.} schools well enough to be able to analyse any problem under each approach.
\end{idea}
 \end{frame}
%%%%%%%%%%%%%%%%%%%%%%%%%%%%%%%%%%
\begin{frame}{So long, and thanks for all the fish!}
 Remember, kids:
 \begin{center}
 {\Huge Bayes rules!} 
 \end{center} 
\end{frame}

%%%%%%%%%%%%%%%%%%%%%%%%%%%%%%%%%%
\begin{frame}{Recommended reading}
\begin{itemize}
  \item[\faBook] \cite{Jaynes1976},~\cite{Efron1986} and Ch 11 of~\cite{Robert2007}.
%  \item
 \end{itemize} 
\end{frame}

\include{lecture_extra}
%%%%%%%
\begin{frame}[t, allowframebreaks]
\frametitle{References}
\bibliographystyle{apalike}
\bibliography{bayes}
\end{frame}
\end{document}
